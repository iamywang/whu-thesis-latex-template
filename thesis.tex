% -----------------------------------------------------------------------------
% Latex Template for Wuhan University Thesis
%
% Ack to Huang Zhenghua (http://aff.whu.edu.cn/huangzh/)
%
% Modified by iamywang
% Date: Jan 29, 2024
% Updated: Apr 13, 2024
% -----------------------------------------------------------------------------

\documentclass[AutoFakeBold=2]{WHUPhd}
\usepackage{gbt7714}
\usepackage{listings}
\usepackage{multirow}
\usepackage{setspace}
\usepackage{titletoc}
\usepackage{pdfpages}

\hypersetup{hidelinks}

\lstset{
language=C++,
showstringspaces=false,
basicstyle=\small\ttfamily,
rulesepcolor=\color{gray},
breaklines=true,
keywordstyle=\color{purple}\bfseries,
commentstyle=\color{gray!80!black},
stringstyle=\color{blue!80!black},
frame=single,
flexiblecolumns=true,
lineskip={-1pt}
}

\begin{document}

% 封面表头信息
\fenleihao{TP309}
\miji{}
\UDC{}
\bianhao{10486}

% 封面内容
\title{安全计算机体系结构关键技术研究}
\author{张\hfill 三\hfill 三}
\Cstudentid{2\hfill 0\hfill 2\hfill 3\hfill 1\hfill 0\hfill 0\hfill 0\hfill 0\hfill 0\hfill 0\hfill 0\hfill 0}
\Csupervisor{李\hfill 四\hfill \hfill 教\hfill 授}
\Cmajor{网\hfill 络\hfill 空\hfill 间\hfill 安\hfill 全}
\Cspeciality{信\hfill 息\hfill 安\hfill 全}
\date{二〇二四年四月}

\Etitle{Research on Key Techniques for Security Computer Architecture}
\Eauthor{ZHANG San}
\Esupervisor{Prof.~LI Si}
\Eschoolname{School of Cyber Science and Engineering}
\Edate{Apr 2024}

\pdfbookmark[0]{封面}{title}
\maketitle

% 英文封面
% -----------------------------------------------------------------------------
% Latex Template for Wuhan University Thesis
%
% Ack to Huang Zhenghua (http://aff.whu.edu.cn/huangzh/)
%
% Modified by iamywang
% Date: Jan 29, 2024
% Updated: Apr 13, 2024
% -----------------------------------------------------------------------------

\thispagestyle{empty}
\renewcommand{\baselinestretch}{1.5}  %下文的行距
\vspace*{0.5cm}

\begin{center}

\zihao{2} \the\Etitle

\vfill

\zihao{4} \the\Eauthor

\end{center}


% 创新点
% -----------------------------------------------------------------------------
% Latex Template for Wuhan University Thesis
%
% Ack to Huang Zhenghua (http://aff.whu.edu.cn/huangzh/)
%
% Modified by iamywang
% Date: Apr 13, 2024
% -----------------------------------------------------------------------------

\chapter*{论文创新点}
\thispagestyle{empty}

本文的创新点主要有以下几个方面:

\begin{itemize}
\item ZeRØ~\cite{ziad2021zero}提出了独特的内存指令和新颖的元数据编码方案来保护代码和数据指针,仅仅只需要微小的微架构变化。ZeRØ在SPEC CPU2017基准上的性能开销为零,VLSI测量显示了低功率和面积开销。
\item No-FAT~\cite{ziad2021no}将内存分配大小(例如malloc大小)作为一个架构特征,来克服传统内存安全方法的许多棘手问题,例如与不安全软件的兼容性和显著的性能下降。No-FAT在SPEC CPU2017基准测试中产生了8\%的开销,VLSI测量显示了低功率和面积开销。
\end{itemize}


% 目录页面样式设置为含有页眉
\pagenumbering{Roman}
\pagestyle{fancy}
\fancyfancy

% 目录页面样式设置为不含有页眉
% \pagenumbering{Roman}
% \pagestyle{oldplain}

% 目录样式
\titlecontents{chapter}[0em]{\bfseries}{\contentspush{\thecontentslabel \quad}}{}{\titlerule*[4pt]{.}\contentspage}
\titlecontents{section}[1em]{}{\contentspush{\thecontentslabel \quad}}{}{\titlerule*[4pt]{.}\contentspage}
\titlecontents{subsection}[2.5em]{}{\contentspush{\thecontentslabel \quad}}{}{\titlerule*[4pt]{.}\contentspage}
\makeatletter
\def\@dotsep{0.5}
\makeatother

% 目录
\pdfbookmark[0]{目录}{toc}
\begin{spacing}{1.39}
\tableofcontents
\end{spacing}
\newpage

% 插图索引
\renewcommand*{\addvspace}[1]{}
\let\oldnumberline\numberline
\renewcommand{\numberline}{\figurename~\oldnumberline}
\makeatletter
\renewcommand*\l@figure{\@dottedtocline{1}{0em}{2.5em}}
\makeatother
\begin{spacing}{1.39}
\listoffigures
\end{spacing}
\newpage

% 表格索引
\renewcommand{\numberline}{\tablename~\oldnumberline}
\makeatletter
\renewcommand*\l@table{\@dottedtocline{1}{0em}{2.5em}}
\makeatother
\begin{spacing}{1.39}
\listoftables
\end{spacing}

% 中英文缩略语对照表
% -----------------------------------------------------------------------------
% Latex Template for Wuhan University Thesis
%
% Ack to Huang Zhenghua (http://aff.whu.edu.cn/huangzh/)
%
% Modified by iamywang
% Date: Jan 29, 2024
% Updated: Feb 23, 2024
% -----------------------------------------------------------------------------

\chapter{中英文缩略语对照表}

\begin{center}
\begin{spacing}{1.39}
\renewcommand\arraystretch{1}
\setlength{\tabcolsep}{4pt}
\begin{tabular}{p{0.1\linewidth}p{0.5\linewidth}p{0.3\linewidth}}
AES & Advanced Encryption Standard & 高级加密标准 \\
API & Application Programming Interface & 应用程序编程接口 \\
ARM & Advanced RISC Machine & 高级精简指令集机器 \\
ASLR & Address Space Layout Randomization & 地址空间布局随机化 \\

BHB & Branch History Buffer & 分支历史缓冲区 \\
BHI & Branch History Injection & 分支历史注入 \\
BPU & Branch Prediction Unit & 分支预测单元 \\
BRB & Branch Retention Buffer & 分支保留缓冲区 \\
BTB & Branch Target Buffer & 分支目标缓冲区 \\

CNN & Convolutional Neural Network & 卷积神经网络 \\
CPU & Central Processing Unit & 中央处理器 \\
CVE & Common Vulnerabilities and Exposures & 通用漏洞披露 \\
\end{tabular}
\end{spacing}
\end{center}


% 摘要
% -----------------------------------------------------------------------------
% Latex Template for Wuhan University Thesis
%
% Ack to Huang Zhenghua (http://aff.whu.edu.cn/huangzh/)
%
% Modified by iamywang
% Date: Jan 29, 2024
% -----------------------------------------------------------------------------

%%%%%%%%%%%%%%%%%%%%%%%
%%% ------------ 中文摘要 ---------------%%%
%%%%%%%%%%%%%%%%%%%%%%%
\begin{cnabstract}
摘要。
\end{cnabstract}
\vspace{1em}\par\vfill

%%%--------- 关键词 -------- %%%
\cnkeywords{侧信道攻击}

%%%%%%%%%%%%%%%%%%%%%%%


%%%%%%%%%%%%%%%%%%%%%%%
%%% ------------ 英文摘要 ---------------%%%
%%%%%%%%%%%%%%%%%%%%%%%
\begin{enabstract}
Abstract.
\end{enabstract}
\vspace{1em}\par\vfill

%%%------ 英文关键词 ------- %%%
\enkeywords{Side-Channel Attack}


% 正文样式
\mainmatter
\renewcommand{\baselinestretch}{1.39}
\renewcommand{\arraystretch}{1.39}
\baselineskip=20pt
\pagestyle{fancy}
\fancyfancy

% 正文
% -----------------------------------------------------------------------------
% Latex Template for Wuhan University Thesis
%
% Ack to Huang Zhenghua (http://aff.whu.edu.cn/huangzh/)
%
% Modified by iamywang
% Date: Jan 29, 2024
% -----------------------------------------------------------------------------

\chapter{绪论}\label{chap1}

\section{研究背景与意义}
随着XXXX\cite{aciiccmez2010new}发展,

XXXX

进一步XXXXXX。

\section{国内外研究现状}
XXXX
\subsection{XXXX技术研究现状}
XXXX
\subsection{XXXX技术研究现状}
XXXX
\subsection{XXXX技术研究现状}


% -----------------------------------------------------------------------------
% Latex Template for Wuhan University Thesis
%
% Ack to Huang Zhenghua (http://aff.whu.edu.cn/huangzh/)
%
% Modified by iamywang
% Date: Jan 29, 2024
% -----------------------------------------------------------------------------

\chapter{相关工作综述}\label{chap2}

% -----------------------------------------------------------------------------
% Latex Template for Wuhan University Thesis
%
% Ack to Huang Zhenghua (http://aff.whu.edu.cn/huangzh/)
%
% Modified by iamywang
% Date: Jan 29, 2024
% -----------------------------------------------------------------------------

\chapter{面向XXXXX的XXXXX}\label{chap3}

\include{chapters/chapter4}
% -----------------------------------------------------------------------------
% Latex Template for Wuhan University Thesis
%
% Ack to Huang Zhenghua (http://aff.whu.edu.cn/huangzh/)
%
% Modified by iamywang
% Date: Jan 29, 2024
% -----------------------------------------------------------------------------

\chapter{基于XXXXX的XXXXX}\label{chap5}

\include{chapters/chapter6}

% 参考文献
\phantomsection
\addcontentsline{toc}{chapter}{参考文献}
\renewcommand{\bibfont}{\zihao{5}}
\setlength{\bibsep}{0pt}
\bibliographystyle{gbt7714-numerical}
\bibliography{bibs/ref1}

% 科研成果和致谢
\backmatter
% -----------------------------------------------------------------------------
% Latex Template for Wuhan University Thesis
%
% Ack to Huang Zhenghua (http://aff.whu.edu.cn/huangzh/)
%
% Modified by iamywang
% Date: Jan 29, 2024
% Updated: Apr 13, 2024
% -----------------------------------------------------------------------------

\reseachresult

\begin{enumerate}[{[}1{]}]
\item 2024 IEEE International Symposium on XX,2024,第一作者
\item IEEE Transactions on XX,2023,第一作者
\item XX学报,2022,第一作者
\item ACM Transactions on XX,2021,第一作者
\end{enumerate}
% -----------------------------------------------------------------------------
% Latex Template for Wuhan University Thesis
%
% Ack to Huang Zhenghua (http://aff.whu.edu.cn/huangzh/)
%
% Modified by iamywang
% Date: Apr 13, 2024
% -----------------------------------------------------------------------------

\acknowledgement


\end{document}
