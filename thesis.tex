% -----------------------------------------------------------------------------
% Latex Template for Wuhan University Thesis
%
% Ack to Huang Zhenghua (http://aff.whu.edu.cn/huangzh/)
%
% Modified by iamywang
% Date: Jan 29, 2024
% -----------------------------------------------------------------------------

% 选项 [forprint]: 交付打印时, 加上此选项, 以消除链接文字之彩色, 避免打印偏淡.
% 选项 [forlib]: 提交给图书馆的电子版, 需要加上选项 forlib, 以消除空白页和彩色链接.
\documentclass[forlib,AutoFakeBold=2]{WHUPhd}
\usepackage{gbt7714}
\usepackage{listings}
\usepackage{multirow}
\usepackage{setspace}
\usepackage{titletoc}
\usepackage{pdfpages}

\hypersetup{hidelinks}

\lstset{
language=C++,
showstringspaces=false,
basicstyle=\small\ttfamily,
rulesepcolor=\color{gray},
breaklines=true,
keywordstyle=\color{purple}\bfseries,
commentstyle=\color{gray!80!black},
stringstyle=\color{blue!80!black},
frame=single,
flexiblecolumns=true,
lineskip={-1pt}
}

\begin{document}

% 封面表头信息
\fenleihao{TP309}
\miji{}
\UDC{}
\bianhao{10486}

% 封面内容
\title{XXX关键技术研究}
\Etitle{Research on Key Techniques for XXXX}
\author{X\hfill X\hfill X}
\Eauthor{XXX}
\Csupervisor{X\hfill X\hfill \hfill 教\hfill 授}
\Esupervisor{Prof.~XX}
\Cmajor{网\hfill 络\hfill 空\hfill 间\hfill 安\hfill 全}
\Emajor{Cyberspace Security}
\Cspeciality{X\hfill X\hfill X\hfill X}
\Especiality{XXXX}
\Schoolname{School of Cyber Science and Engineering}
\date{二〇二四年一月}
\Edate{Jan 2024}

\pdfbookmark[0]{封面}{title}
\maketitle

% 英文封面
% -----------------------------------------------------------------------------
% Latex Template for Wuhan University Thesis
%
% Ack to Huang Zhenghua (http://aff.whu.edu.cn/huangzh/)
%
% Modified by iamywang
% Date: Jan 29, 2024
% -----------------------------------------------------------------------------

\thispagestyle{empty}
\renewcommand{\baselinestretch}{1.5}  %下文的行距
\vspace*{0.5cm}

\begin{center}{\zihao{2} \the\Etitle \par}\end{center}

\vfill

\begin{center}
\zihao{4}
% 表格第二列0.5cm
\setlength{\tabcolsep}{0pt}
\begin{tabular}{rcl}
\makebox{Candidate}  & \makebox[0.5cm][c]{:}  & \makebox{\the\Eauthor}      \\
\makebox{Supervisor} & \makebox[0.5cm][c]{:}  & \makebox{\the\Esupervisor}  \\
\makebox{Major}      & \makebox[0.5cm][c]{:}  & \makebox{\the\Emajor}       \\
\makebox{Speciality} & \makebox[0.5cm][c]{:}  & \makebox{\the\Especiality}
\end{tabular}

\vspace*{2cm}
\begin{center}
  \iflib % 向图书馆提交电子文档, 使用黑白校徽.
  \includegraphics[height=4cm]{whulogo.eps}       %%  黑白的. 很小, 只有 10k.
  \else
  \ifprint % 文档打印, 使用黑白校徽.
  \includegraphics[height=4cm]{whulogo.eps}       %%  黑白的.
  \else
  \includegraphics[height=4cm]{whulogo.eps}       %%  彩色的.
  \fi
  \fi
\end{center}


\zihao{-2}
\the\Schoolname\\
{WUHAN UNIVERSITY}

\vspace*{1.0cm}

\the\Edate

\end{center}
%%%%%%%--判断是否需要空白页-----------------------------
  \iflib
  \else
  \newpage
  \thispagestyle{empty}
  \cleardoublepage
  \fi
%%%%%%%-------------------------------------------------
{\pagestyle{empty}
%%%--- 加入``郑重声明'' --- %%%%%%%%%%%%%%%%%
\input{includes/原创性声明}%%%%%%%%%%%%%%%%%
%%% ---加入``武汉大学学位论文使用授权协议书'' ---  %%%%%%
% -----------------------------------------------------------------------------
% Latex Template for Wuhan University Thesis
%
% Ack to Huang Zhenghua (http://aff.whu.edu.cn/huangzh/)
%
% Modified by iamywang
% Date: Jan 29, 2024
% Updated: Apr 13, 2024
% -----------------------------------------------------------------------------

\newpage
\vspace*{20pt}
\thispagestyle{empty}
\begin{center}{\zihao{4}\heiti \bfseries 武汉大学学位论文使用授权协议书}\end{center}

\vspace*{4pt}

\begin{center}{\zihao{-5}(一式两份,一份论文作者保存,一份留学校存档)}\end{center}

\vspace*{4pt}

本学位论文作者愿意遵守武汉大学关于保存、使用学位论文的管理办法及规定,
即:学校有权保存学位论文的印刷本和电子版,并提供文献检索与阅览服务;
学校可以采用影印、缩印、数字化或其它复制手段保存论文;
在以教学与科研服务为目的前提下,学校可以在校园网内公布部分或全部内容。

一、在本论文提交当年,同意在校园网内以及中国高等教育文献保障系统(CALIS)、高校学位论文系统提供查询及前十六页浏览服务。

二、在本论文提交~$\Box$~当年/~$\Box$~一年/~$\Box$~两年/~$\Box$~三年以后,
同意在校园网内允许读者在线浏览并下载全文,学校可以为存在馆际合作关系的兄弟高校用户提供文献传递服务和交换服务。
(保密论文解密后遵守此规定)

\vskip 0.8cm

论文作者(签名):\raisebox{-1ex}{\underline{\makebox[3.54cm][c]{}}}
\vskip 0.8cm

学\qquad 号:\raisebox{-1ex}{\underline{\makebox[5cm][c]{}}}
\vskip 0.8cm

学\qquad 院:\raisebox{-1ex}{\underline{\makebox[5cm][c]{}}}

\vskip 1cm
\begin{flushright}
日期:\hskip2cm 年\hskip1.2cm 月\hskip1.2cm 日
\end{flushright}

%%%%%%%%%%%%%%%%%%
%%%%%%%%%%%%%%%%%%%%%%%%%%%%%%%
}

%%%%%%%%%%%%%%%%%%%%%%%%%%%%%%%
%%% ------------------- 论文创新点----------------------- %%%
%%%%%%%%%%%%%%%%%%%%%%%%%%%%%%%

\newpage\vspace*{20pt}\thispagestyle{empty}
\begin{center}{\zihao{-2}\heiti 论文创新点}\end{center}
\par\vspace*{30pt}
\baselineskip=20pt

%%%%%%%%%%%%%%%%%%%%%%%%%%%%%%%
%%%%%%%--请勿删除以下内容 -------------------------------%
%%%%%%%--判断是否需要空白页-----------------------------%
  \iflib
  \let\cleardoublepage\clearpage
  \else
  \newpage
  \thispagestyle{empty}
  \cleardoublepage
  \fi
%%%%%%%---------------------------------------------------%


% 目录页面样式设置为含有页眉
\frontmatter
\pagenumbering{Roman}
\pagestyle{fancy}
\fancyfancy

% 目录页面样式设置为不含有页眉
% \frontmatter
% \pagenumbering{Roman}
% \pagestyle{oldplain}

% 目录样式
\titlecontents{chapter}[0em]{\bfseries}{\contentspush{\thecontentslabel \quad}}{}{\titlerule*[4pt]{.}\contentspage}
\titlecontents{section}[1em]{}{\contentspush{\thecontentslabel \quad}}{}{\titlerule*[4pt]{.}\contentspage}
\titlecontents{subsection}[2.5em]{}{\contentspush{\thecontentslabel \quad}}{}{\titlerule*[4pt]{.}\contentspage}
\makeatletter
\def\@dotsep{0.5}
\makeatother

% 目录
\pdfbookmark[0]{目录}{toc}
\begin{spacing}{1.39}
\tableofcontents
\end{spacing}
\newpage

% 插图索引
\renewcommand*{\addvspace}[1]{}
\let\oldnumberline\numberline
\renewcommand{\numberline}{\figurename~\oldnumberline}
\makeatletter
\renewcommand*\l@figure{\@dottedtocline{1}{0em}{2.5em}}
\makeatother
\begin{spacing}{1.39}
\listoffigures
\end{spacing}
\newpage

% 表格索引
\renewcommand{\numberline}{\tablename~\oldnumberline}
\makeatletter
\renewcommand*\l@table{\@dottedtocline{1}{0em}{2.5em}}
\makeatother
\begin{spacing}{1.39}
\listoftables
\end{spacing}
\cleardoublepage

% 中英文缩略语对照表
% -----------------------------------------------------------------------------
% Latex Template for Wuhan University Thesis
%
% Ack to Huang Zhenghua (http://aff.whu.edu.cn/huangzh/)
%
% Modified by iamywang
% Date: Jan 29, 2024
% Updated: Feb 23, 2024
% -----------------------------------------------------------------------------

\chapter{中英文缩略语对照表}

\begin{center}
\begin{spacing}{1.39}
\renewcommand\arraystretch{1}
\setlength{\tabcolsep}{4pt}
\begin{tabular}{p{0.1\linewidth}p{0.5\linewidth}p{0.3\linewidth}}
AES & Advanced Encryption Standard & 高级加密标准 \\
API & Application Programming Interface & 应用程序编程接口 \\
ARM & Advanced RISC Machine & 高级精简指令集机器 \\
ASLR & Address Space Layout Randomization & 地址空间布局随机化 \\

BHB & Branch History Buffer & 分支历史缓冲区 \\
BHI & Branch History Injection & 分支历史注入 \\
BPU & Branch Prediction Unit & 分支预测单元 \\
BRB & Branch Retention Buffer & 分支保留缓冲区 \\
BTB & Branch Target Buffer & 分支目标缓冲区 \\

CNN & Convolutional Neural Network & 卷积神经网络 \\
CPU & Central Processing Unit & 中央处理器 \\
CVE & Common Vulnerabilities and Exposures & 通用漏洞披露 \\
\end{tabular}
\end{spacing}
\end{center}


% 摘要
% -----------------------------------------------------------------------------
% Latex Template for Wuhan University Thesis
%
% Ack to Huang Zhenghua (http://aff.whu.edu.cn/huangzh/)
%
% Modified by iamywang
% Date: Jan 29, 2024
% -----------------------------------------------------------------------------

%%%%%%%%%%%%%%%%%%%%%%%
%%% ------------ 中文摘要 ---------------%%%
%%%%%%%%%%%%%%%%%%%%%%%
\begin{cnabstract}
摘要。
\end{cnabstract}
\vspace{1em}\par\vfill

%%%--------- 关键词 -------- %%%
\cnkeywords{侧信道攻击}

%%%%%%%%%%%%%%%%%%%%%%%


%%%%%%%%%%%%%%%%%%%%%%%
%%% ------------ 英文摘要 ---------------%%%
%%%%%%%%%%%%%%%%%%%%%%%
\begin{enabstract}
Abstract.
\end{enabstract}
\vspace{1em}\par\vfill

%%%------ 英文关键词 ------- %%%
\enkeywords{Side-Channel Attack}


% 正文样式
\mainmatter
\renewcommand{\baselinestretch}{1.5}
\renewcommand{\arraystretch}{1.5}
\baselineskip=20pt
\CTEXsetup[nameformat={\normalfont\zihao{-2}\heiti\raggedright},
titleformat={\normalfont\zihao{-2}\heiti\raggedright},
number={\arabic{chapter}},name={,},afterskip={4pt},beforeskip={-28pt}]{chapter}
\pagestyle{fancy}
\fancyfancy

% 正文
% -----------------------------------------------------------------------------
% Latex Template for Wuhan University Thesis
%
% Ack to Huang Zhenghua (http://aff.whu.edu.cn/huangzh/)
%
% Modified by iamywang
% Date: Jan 29, 2024
% -----------------------------------------------------------------------------

\chapter{绪论}\label{chap1}

\section{研究背景与意义}
随着XXXX\cite{aciiccmez2010new}发展,

XXXX

进一步XXXXXX。

\section{国内外研究现状}
XXXX
\subsection{XXXX技术研究现状}
XXXX
\subsection{XXXX技术研究现状}
XXXX
\subsection{XXXX技术研究现状}


% -----------------------------------------------------------------------------
% Latex Template for Wuhan University Thesis
%
% Ack to Huang Zhenghua (http://aff.whu.edu.cn/huangzh/)
%
% Modified by iamywang
% Date: Jan 29, 2024
% -----------------------------------------------------------------------------

\chapter{相关工作综述}\label{chap2}

% -----------------------------------------------------------------------------
% Latex Template for Wuhan University Thesis
%
% Ack to Huang Zhenghua (http://aff.whu.edu.cn/huangzh/)
%
% Modified by iamywang
% Date: Jan 29, 2024
% -----------------------------------------------------------------------------

\chapter{面向XXXXX的XXXXX}\label{chap3}

\include{chapters/chapter4}
% -----------------------------------------------------------------------------
% Latex Template for Wuhan University Thesis
%
% Ack to Huang Zhenghua (http://aff.whu.edu.cn/huangzh/)
%
% Modified by iamywang
% Date: Jan 29, 2024
% -----------------------------------------------------------------------------

\chapter{基于XXXXX的XXXXX}\label{chap5}

\include{chapters/chapter6}
% -----------------------------------------------------------------------------
% Latex Template for Wuhan University Thesis
%
% Ack to Huang Zhenghua (http://aff.whu.edu.cn/huangzh/)
%
% Modified by iamywang
% Date: Jan 29, 2024
% -----------------------------------------------------------------------------

\chapter{总结与展望}\label{chap7}

\section{本文总结}

\section{未来工作展望}


% 参考文献
\cleardoublepage\phantomsection
\addcontentsline{toc}{chapter}{参考文献}
\renewcommand{\bibfont}{\zihao{5}}
\setlength{\bibsep}{0pt}
\bibliographystyle{gbt7714-numerical}
\bibliography{bibs/ref1, bibs/ref2, bibs/ref3, bibs/ref4, bibs/ref5, bibs/ref6, bibs/ref7}

% 科研成果和致谢
\backmatter
% -----------------------------------------------------------------------------
% Latex Template for Wuhan University Thesis
%
% Ack to Huang Zhenghua (http://aff.whu.edu.cn/huangzh/)
%
% Modified by iamywang
% Date: Jan 29, 2024
% -----------------------------------------------------------------------------

\reseachresult

\subsection*{作者简历}

XXX(19XX—),男,湖北武汉人,武汉大学XXX学院博士研究生

研究方向:XXXX

2021年9月——至今,武汉大学,XXXX

\subsection*{学术论文}
\begin{enumerate}[{[}1{]}]

\item \textbf{XXX}, XXX*.
Title.
\textit{Jounral/Conference}\\
\textbf{(CCF-A类会议,对应论文第~\ref{chap3}~章)}
\end{enumerate}

\subsection*{发明专利}
\begin{enumerate}[{[}1{]}]
\item 一种XXXX方法及系统,
发明人:XXX、\textbf{XXX},申请中,
申请号:XXXXX,申请日:XXXX年XX月XX日
\end{enumerate}

\subsection*{获奖情况}
\begin{enumerate}[{[}1{]}]
\item 武汉大学XXXX奖学金,一等奖,2021
\item 武汉大学XXXX奖学金,一等奖,2022
\end{enumerate}

\acknowledgement

%%%%%%%%%%%%%%%%%%%%%%%%%%%%%%%%%%%%%%%
%%%%%%%--判断是否需要空白页-----------------------------
  \iflib
  \else
  \newpage
  %\thispagestyle{empty}`
  \cleardoublepage
  \fi
%%%%%%%-------------------------------------------------


\cleardoublepage
\end{document}
