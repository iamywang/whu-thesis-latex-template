% -----------------------------------------------------------------------------
% Latex Template for Wuhan University Thesis
%
% Ack to Huang Zhenghua (http://aff.whu.edu.cn/huangzh/)
%
% Modified by iamywang
% Date: Jan 29, 2024
% Updated: Apr 13, 2024
% -----------------------------------------------------------------------------

\chapter{绪论}\label{chap1}

2018年1月3日, Google Project Zero团队的Jann Horn等安全研究者公开了两组处理器漏洞,
即Meltdown漏洞~\cite{Lipp2018meltdown}和Spectre漏洞~\cite{kocher2019spectre}。
其中Meltdown对应的漏洞编号为CVE-2017-5754(流氓数据缓存加载),
这种攻击“熔化”了由硬件来实现的安全边界,允许用户级别的应用程序“越界”访问系统级的内存,从而造成数据泄露。
而Spectre对应的漏洞编号为CVE-2017-5753(边界检查绕过)和CVE-2017-5715(分支目标注入),
利用分支预测的错误推测,让攻击者有能力触发受害者访问特定的敏感数据,并通过隐蔽信道泄露信息。

不管是Meltdown攻击还是Spectre攻击,其本质都是利用现代处理器的优化策略,
而这些优化策略在发生错误时,并不会造成架构层面(或者说ISA层面)上数据的变化,
但是由于现代处理器引入了cache、TLB、缓冲区等微架构元素,攻击者仍然有能力通过观测微架构的变化来获取信息。
而这些利用乱序执行和分支预测等机制,触发处理器执行错误的指令,
从而造成秘密数据泄露的方式统称为瞬态执行攻击(Transient Execution Attack)~\cite{xiong2021survey}。
