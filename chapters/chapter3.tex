% -----------------------------------------------------------------------------
% Latex Template for Wuhan University Thesis
%
% Ack to Huang Zhenghua (http://aff.whu.edu.cn/huangzh/)
%
% Modified by iamywang
% Date: Jan 29, 2024
% Updated: Apr 13, 2024
% -----------------------------------------------------------------------------

\chapter{内存安全问题研究}\label{chap3}

从计算机体系结构的角度来说,内存安全违规(memory safety violation)一般分为两大类:

(1)时间违规(temporal violation):典型代表为Use-After-Free(UAF)漏洞,即在释放内存后,再次对该内存进行访问。

(2)空间违规(spatial violation):典型代表为栈溢出(stack overflow)漏洞,即在栈上分配的内存空间不足以存放当前的数据。

而面向返回编程(Return-Oriented Programming,ROP)、面向跳转编程(Jump-Oriented Programming,JOP)和面向调用编程(Call-Oriented Programming,COP)都是基于空间违规的攻击手段,本文将从攻击基本原理、现有芯片级对策和思考三个方面来介绍ROP/JOP/COP。

而ROP攻击产生的一大原因是因为,现代操作系统为了防止攻击者在栈上发起代码注入漏洞,采用了$W \oplus X$的内存保护机制,即栈上的内存空间只能执行,不能写入。因此,攻击者在栈上发起代码注入漏洞时,只能通过覆盖返回地址来控制程序的执行流程,而这种攻击手段就是ROP。